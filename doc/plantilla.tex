% !TEX TS-program = pdflatex
% !TEX encoding = UTF-8 Unicode

% This is a simple template for a LaTeX document using the "article" class.
% See "book", "report", "letter" for other types of document.

\documentclass[11pt]{article} % use larger type; default would be 10pt

\usepackage[utf8]{inputenc} % set input encoding (not needed with XeLaTeX)

%%% Examples of Article customizations
% These packages are optional, depending whether you want the features they provide.
% See the LaTeX Companion or other references for full information.

%%% PAGE DIMENSIONS
\usepackage{geometry} % to change the page dimensions
\geometry{a4paper} % or letterpaper (US) or a5paper or....
% \geometry{margin=2in} % for example, change the margins to 2 inches all round
% \geometry{landscape} % set up the page for landscape
%   read geometry.pdf for detailed page layout information

\usepackage{graphicx} % support the \includegraphics command and options

% \usepackage[parfill]{parskip} % Activate to begin paragraphs with an empty line rather than an indent

%%% PACKAGES
\usepackage{booktabs} % for much better looking tables
\usepackage{array} % for better arrays (eg matrices) in maths
%\usepackage{paralist} % very flexible & customisable lists (eg. enumerate/itemize, etc.)
\usepackage{verbatim} % adds environment for commenting out blocks of text & for better verbatim
\usepackage{subfig} % make it possible to include more than one captioned figure/table in a single float
% These packages are all incorporated in the memoir class to one degree or another...

%%% HEADERS & FOOTERS
\usepackage{fancyhdr} % This should be set AFTER setting up the page geometry
\pagestyle{fancy} % options: empty , plain , fancy
\renewcommand{\headrulewidth}{0pt} % customise the layout...
\lhead{}\chead{}\rhead{}
\lfoot{}\cfoot{\thepage}\rfoot{}

%%% SECTION TITLE APPEARANCE
\usepackage{sectsty}
\allsectionsfont{\sffamily\mdseries\upshape} % (See the fntguide.pdf for font help)
% (This matches ConTeXt defaults)

%%% ToC (table of contents) APPEARANCE
\usepackage[nottoc,notlof,notlot]{tocbibind} % Put the bibliography in the ToC
\usepackage[titles,subfigure]{tocloft} % Alter the style of the Table of Contents
\renewcommand{\cftsecfont}{\rmfamily\mdseries\upshape}
\renewcommand{\cftsecpagefont}{\rmfamily\mdseries\upshape} % No bold!

%%% END Article customizations

\usepackage[spanish]{babel}
\usepackage{listings} 
%%% The "real" document content comes below...

\title{Investigación de Lenguajes - Scilab}
\author{Gabriel Aumala, Wilson Enriquez}
%\date{} % Activate to display a given date or no date (if empty),
         % otherwise the current date is printed 

\begin{document}
\maketitle
%\tableofcontents % No hace falta un TOC en un artículo corto

\section{Introducción}
Interpretar nuestras ideas matemáticas no es una tarea sencilla para muchas personas, mucho menos para una computadora. Scilab es un lenguaje de programación de alto nivel que se creó para servir de intérprete a la hora de necesitar la ayuda de una computadora para realizar cálculos científicos complicados.  Creado en Enero de 1994, Scilab contiene cientos de funciones matemáticas aplicadas comúnmente en las ciencias junto a las estructuras de datos y gráficos en 2D y 3D encontrados en un lenguaje de alto nivel. Muchos cálculos rudimentarios pueden ser resueltos en Scilab con pocas líneas de código, a diferencia de otros lenguajes que podrían necesitar funciones y librerías adicionales para el mismo problema. En la actualidad, Scilab es usado ampliamente por instituciones tanto de educación secundaria como de educación superior para enseñar varias asignaturas necesarias para la ingeniería y las ciencias matemáticas. Scilab es un software fácil de instalar, gratuito y de código abierto ya que está financiado por Scilab Enterprises. 

\begin{figure}
  \centering
    \includegraphics{logosci}
  \caption{Logo oficial de Scilab}
  \label{fig:ejemplo}
\end{figure}

\section{Características}
\section{Historia}

\section{Tutorial de Instalación}
1.	Comenzamos entrando a la página web de Scilab usando la URL: http://www.scilab.org/

\begin{figure}
  \centering
    \includegraphics{Captura1}
  \caption{Paso 1}
  \label{fig:paso1}
\end{figure}

2.	Hacemos click en la flecha que dice “Download Scilab” y procedemos a guardar el instalador en nuestro ordenador.

3.	Procedemos a abrir el instalador.

\begin{figure}
  \centering
    \includegraphics{Captura2}
  \caption{Paso 2}
  \label{fig:paso2}
\end{figure}

4.	Seleccionamos siguiente, luego aceptamos el acuerdo de licencia y elegimos siguiente.

\begin{figure}
  \centering
    \includegraphics{Captura3}
  \caption{Paso 3}
  \label{fig:paso3}
\end{figure}

5.	Elegimos la carpeta de destino donde se desea instalar Scilab y hacemos click en siguiente.

\begin{figure}
  \centering
    \includegraphics{Captura4}
  \caption{Paso 4}
  \label{fig:paso4}
\end{figure}

6.	Elegimos “Installation (Default)” y escogemos siguiente.

\begin{figure}
  \centering
    \includegraphics{Captura5}
  \caption{Paso 5}
  \label{fig:paso5}
\end{figure}

7.	Seleccionamos los iconos de acceso directo que deseamos agregar y damos click en siguiente.

\begin{figure}
  \centering
    \includegraphics{Captura6}
  \caption{Paso 6}
  \label{fig:paso6}
\end{figure}

8.	Procedemos a instalar Scilab.

\section{Hola Mundo y otros Programas Introductorios}

\lstset{language=Pascal}          % Set your language (you can change the language for each code-block optionally)

\begin{lstlisting}[frame=single]  % Start your code-block
for i:=maxint to 0 do
begin
{ do nothing }
end;
Write('Case insensitive ');
Write('Pascal keywords.');
\end{lstlisting}



\end{document}